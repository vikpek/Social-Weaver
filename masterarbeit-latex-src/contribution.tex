\section{Contribution}\label{contribution}

In the last couple of years the internet developed into a mass medium. It started with with the simple asynchronous one-to-one communication such as E-mail. Today we have all kinds of communication types: forums and bulletin boards support many-to-many information exchange, one-to-many is being provided by services like Twitter, chats or instant message give us the possibility of synchronous transmissions. With the launch of webcams the internet took regular voice calls to another level adding the opportunity to actually see each other. Success stories like Facebook and Twitter show us, that the way of how human communicate are still in development. The problem is that we see communication as something we were practicing since we exist. But the internet offers us new possibilities therefore we need to take another perspective on communication. 

The literal language, as we know it, is powerful. In fact so powerful, that teaching machines to speak and understand is still one of the greater challenges. In brings great advantages for communication. In case we cannot remember a specific word, it is easy for us to come up with an alternative or to somehow outline it. Even persons that are not speaking the same language, will be able to somehow communicate with each other using gestures or images. But on the other hand literal language has its shortcomings. To describe technical or scientific topics precisely we need a lot of words to bring it into understandable context. Everyone who sat in lecture that was a bit over his skills exactly knows this problem to well. The more concrete and complex something becomes, the more we feel the shortcomings of literal expressions. 

Software makes no exception. Applications are built while keeping in mind, that a user will actually see the interface and interact with it. We are using a button because a user sees it and pushes it. Where the button is located or in which context it has which functionality is obvious to the user. At least it should be. But what if he wants to discuss something about this button with his colleagues? This could be a question or criticism. Nevertheless he will need to describe where the button is; in which workflow it appears and so on. The usual way would be to create a screenshot, write an explanation, compose an E-mail and send it. From there on the E-mail thread would become the central discussion point related to our button. This is not optimal at all! What if other colleagues might have something to add to the conversation, but are not in included in the receivers list? If it is just a short question, the way through a screenshot etc. is not time efficient and exhausting for all participants. What if the information in the E-mail thread might be informative for other users in future? The would have to ask the same question again. 

So what we want is the possibility to create some form of communication functionality - directly related to the button. And which is visible to a group of users for optionally unlimited amount of time.

Well a comment box beneath the button would solve the problem. Or a link that leads to a discussion forum - just for the button. But besides that both solutions bring a bunch of disadvantages, it would require to modify the web application that includes our favorite button. So lets drop these possibilities. What if we would have the opportunity to inject social elements directly into our web application without the need to modify it. Basically web applications run in a browser and what is displayed can be modified locally. And that is what we do. We weave social elements into the browser view and synchronize it for different user sessions. This way we reach exactly the functionality we need to solve our problem without touching the web application. We call this process Social Weaving. 

In the following we are first going to discuss the idea of Social Weaving based on  an abstract requirements analysis of a prototype. What functionality does it need to achieve the goals we mentioned above and what are the difficulties? In the second part we are going down on a concrete level where we take the theory from the first part into action and actually explain the architecture and implementation of the prototype, Social Weaver, in detail. 